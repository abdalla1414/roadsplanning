\documentclass{article}
\usepackage[utf8]{inputenc}

\title{roads planning}
\author{aaabdallah.com }
\date{February 2021}

\begin{document}

\maketitle

\section{Introduction}

\subsection{{shortest path detection by using Dijkestra }}


In this paper \cite{anam2019android} researcher conduct the study which build an Android app
to help the tourist find Besides historical sites like Kraton, Traditional Art and Cultural and show the closest distance between  places  in Sumenep  in the eastern end of Madura Island.This research builds an Android-based GIS using the Dijkstra Algorithm, which is supposed to help tourist in-display directions from one place to another in the city of Sumenep, But the main problem is the app doesn't  predicts traffic flow to  finding the best route between two locations on the road network during rush hours just show the closest distance between  places .


In \cite{iskandar2019development} the authors developed a phone application that can help tourists determine the shortest route to finding the best route between two tourist attractions locations on the road network to avoid wasting time on trips. by using the Java programming language to make the application can be run on many computers and smartphones.
 And By using Dijkstra's algorithm, Tourist places are determined as a node Then seek the distance from the first node to the closest node one by one .The short paths created by the application are shorter than the ones recommended by Google Maps but In some places the results are close due to other factors.

\subsection{ shortest path detection by Using Bellman-Ford Algorithm}



In this paper\cite{arrumdany2019web} Arrumdany et al enhanced GIS-based network analysis was
performed and applied to the Small and Medium Enterprises (MSME) because of Tourist's ignorance of MSME and their products And another problem is to find these small Enterprises.
Therefore, the authors developed an application  for MSME to launch their businesses and display all information about the MSME business in Medan City,Indonesian such as a geographical
 information system that contains a mapping of the shortest MSME roads, estimated distances and driving time to these MSME. This app uses the Bellman-Ford algorithm and Euclidean
Distance method to map the shortest roads to MSME.Because there aren't many locations, the system has the accuracy of determining the shortest path to the many locations is still less than optimal.













\subsection{shortest path detection by using Haversine Formula}


in this paper\cite{ingole2013landmark} Ingole et al enhanced GIS-based network analysis was performed and applied to the Indian citiesroad network. To help tourists because most of the time, the tourist cannot find even the important , close tourist places in the city, because  due to the lack of naming boards and other significant.The author chose Haversine formula and Dijkstra’s algorithm because  these methods is used as solution of problems related to distance calculation and calculate  shortest path between places.The tourist was given the shortest route with directions and appropriate turns that direct the user to visit  Most of  landmarks.


In Chopde's (2013) study\cite{chopde2013landmark} using the A * algorithm to find the path shortest between two Landmarks and haversine formula for calculating  Landmarks location distance 
 this paper focuses on giving an appropriate path to the user by directing them through many junctions and routes that can be easily identified through the associated places and Google Map.
The shortest path was found in two ways: a text path and a graphical method:
The user can know the shortest path from the beginning to the destination in a text-based path or a graphical path using Google Maps
\subsection{shortest path detection by using Other Application }
Shyty and Kushi (2012)\cite{shyti2012impact} consider that if people used the GIS systems in the tourism business, it's important for all choices that a tourist or an investor takes in order to plan a trip, or respectively to make a tourist investment. The authors tried to solve the problem Elbasan region is one of the biggest towns in Albania and a major tourist destination and have many tourist sites, but there are limited ways and means of transportation to reach these sites is very difficult where there are many difficulties around promotion, such as the lack of graphic tourist maps, the absence of information based on Internet and social media, the absence of digital information for tourist facilities, and insufficient Promotion from the government.In this research, ArcView GIS software shows classified the tourism places in this map (roads, hotels, etc.) and with using ArcGIS 9.2  Conversion of all the old maps to digital maps and creates attribute data and spatial data. Because of a loss of information, statistics, and market research, it is difficult to find any statistics about the number of tourists.
\\

 In this paper Debashis  et al \cite{das2019road} They counted all the necessary roads connecting in the largest city in the Indian state of Assam, Guwahati, in a GIS. To help Population to arrive at their destination with ease and decreased cost and time. The National Highways in India that is owned by the Ministry of Road Transport and Highways is one of the busiest road networks in the world, also deluge road by floodwater and water-logging problems in the Guwahati area during the rainy season.The authors Using GIS capabilities in ArcGIS to the networks analysis of Guwahati City,a roads was added in ArcGIS app from the Google Earth satellite and the layers were being classified as main road and secondary road. The results length of roads  1298.834 km and the shortest roads by Ganesh =12 km, by Fayez  Road=9.3 km and  by Dhirenpara Road=8.9 km.
 \\

In this paper  (Gill and Bharath , 2013)\cite{gill2013identification} used GIS In the tourism and hospitality industry to help tourists find the nearest motels and restaurants and find the shortest and quickest roads to their destination of choice and have detailed and consummate information about those tourist places in Delhi city.
 They used the ArcGIS software and ERDAS software was used for the analysis and database creation and Indian Remote Sensing Satellite (IRSS) data was used to get the road network and tourist places .
 the authors found that if time is used as impedance it covers 87.3 km traveled distance but if the length is used as impedance it covers 85 km traveled distance. 
 \clearpage
\bibliography{ref}
 \bibliographystyle{plain}
\end{document}
